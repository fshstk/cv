% Fabian Hummel – CV
% https://github.com/fshstk/cv
%
% This document uses the Yet Another Awesoe CV class by Christophe Roger:
% https://github.com/darwiin/yaac-another-awesome-cv
%
% Document license:
% CC BY-SA 4.0 (https://creativecommons.org/licenses/by-sa/4.0/)

\sectionTitle{%
    \german{Projekte (\textit{Hardware})}%
    \english{Projects (\textit{Hardware})}%
}{\faMicrochip}
\begin{projects}
    \project
    {midiPAD}
    {2019}
    {
        % no link
    }
    {%
        \german{%
            Bachelorarbeit. Elektronisches Drumpad-Modul, das das Ausgangssignal
            eines piezobasierten Erschütterungssensors in ein entsprechendes
            MIDI Event konvertiert, und über eine USB Schnittstelle sendet.
            Verwendet wird dafür eine analoge Signalverarbeitungsstufe und ein
            ATmega8 Chip auf einem selbst entwickelten PCB, verpackt in einem 3D
            gedruckten Gehäuse.
        }
        \english{%
            BSc project. An electronic drum pad module that converts the output
            signal of a piezo-based percussive sensor and emits a corresponding
            MIDI event via USB interface. The project includes an analog signal
            conditioning circuit and an ATmega8 chip on a custom printed circuit
            board, inside of a 3D printed enclosure.
        }
    }
    {}

    \project
    {Universal USB Prototyping Board}
    {2019}
    {
        % no link
    }
    {%
        \german{%
            Im Rahmen der Bachelorarbeit entwickelt. Arduino-ähnliches
            Mikrocontroller-Board basierend auf dem ATmega8, mit dem es möglich
            ist, USB Geräte zu entwickeln.
        }
        \english{%
            Developed in conjunction with the midiPAD. An ATmega8-based
            prototyping board, similar to Arduino, that enables the user to
            develop USB devices.
        }
    }
    {}

    \project
    {Wetter Ei}
    {2016}
    {
        % no link
    }
    {%
        \german{%
            ESP8266-basierte IoT Wetterstation. Aktualisiert regelmäßig
            Temperatur- und Luftfeuchtigkeitswerte, die über ein Web Interface
            einsehbar sind.
        }
        \english{%
            ESP8266-based IoT weather station. Regularly reads in temperature
            and humidity values that can then be accessed via a web interface.
        }
    }
    {}

\end{projects}