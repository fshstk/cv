% Fabian Hummel – CV
% https://github.com/fshstk/cv
%
% This document uses the Yet Another Awesoe CV class by Christophe Roger:
% https://github.com/darwiin/yaac-another-awesome-cv
%
% Document license:
% CC BY-SA 4.0 (https://creativecommons.org/licenses/by-sa/4.0/)

\sectionTitle{%
	\german{Berufserfahrung}%
	\english{Work Experience}%
}{\faWrench}

\begin{experiences}
	\experience
	{\german{Aktuell}\english{Present}}
	{%
		\german{Wissenschaftlicher Mitarbeiter}%
		\english{Research assistant}%
	}
	{\link{http://www.transcultural.at/}{Transcultural Campaigning}}
	{\german{Wien}\english{Vienna}}
	{2019}
	{
		\german{%
			\begin{itemize}
				\item Recherche und unterstützende Forschungsarbeit im Bereich
				      interkultureller Kommunikation
				\item Betreuung der Webseite
			\end{itemize}
		}
		\english{%
			\begin{itemize}
				\item Research focused on intercultural communication
				\item Website maintenance
			\end{itemize}
		}
		\smallskip
	}
	{%
		\german{Politisches Bewusstsein}\english{political awareness},
		\german{Globale Perspektive}\english{global perspective}
	}

	\emptySeparator

	\experience
	{\german{Aktuell}\english{Present}}
	{%
		\german{Übersetzer}%
		\english{Translator}%
	}
	{%
		\german{Selbstständig}%
		\english{Freelance}%
	}
	{\german{Wien/Graz}\english{Vienna/Graz}}
	{2011}
	{
		\german{%
			\begin{itemize}
				\item Übersetzung deutschsprachiger Texte ins Englische
				\item Korrekturlesen englischsprachiger Texte
				\item Fokus auf wissenschaftliches und technisches Textmaterial
			\end{itemize}
		}
		\english{%
			\begin{itemize}
				\item Translation of German works into English
				\item Proofreading of English works
				\item Particular focus on scientific and technical material
			\end{itemize}
		}
		\smallskip
	}
	{
		\german{Selbstständigkeit}\english{independence},
		\german{Sprachkompetenz}\english{language proficiency},
		\german{Auge für Details}\english{eye for detail}
	}

	\emptySeparator

	\experience
	{\german{Aktuell}\english{Present}}
	{%
		\german{Musikproduzent}%
		\english{Music producer}%
	}
	{%
		\german{Selbstständig}%
		\english{Freelance}%
	}
	{\german{Wien/Graz}\english{Vienna/Graz}}
	{2010}
	{%
		\german{Musik, Jingles, Voice-Overs und Aufnahmen, u.A. für:}
		\english{Music, jingles, voice-overs and recordings for various clients including:}
		\begin{itemize}
			\item Demner, Merlicek \& Bergmann
			\item Arbeiterkammer Wien
			\item Made Jour Label
			\item Stella Models
			\item Edition A
			\item Vöslauer
			\item OMV
		\end{itemize}
		\smallskip
	}
	{
		\german{Kreativität}\english{creativity},
		\german{Leidenschaft}\english{passion},
		\german{Zuverlässigkeit}\english{reliability}
	}

	\emptySeparator

	\experience
	{2019}
	{%
		\german{Wissenschaftlicher Mitarbeiter}%
		\english{Telescope operator}%
	}
	{%
		\german{Österr. Akademie der Wissenschaften}%
		\english{Austrian Academy of Sciences}%
	}
	{Graz}
	{2017}
	{
		\german{%
			\begin{itemize}
				\item Projektmitarbeiter am Institut für Weltraumforschung
				\item \textit{Satellite Laser Ranging} –
				      Nachtmessungen von Satelliten mittels Laserteleskop
			\end{itemize}
		}
		\english{%
			\begin{itemize}
				\item Team member at the space research institute
				\item \textit{Satellite laser ranging} –
				      night-time satellite distance measurements using a laser telescope
			\end{itemize}
		}
		\smallskip
	}
	{
		\german{Ausdauer}\english{endurance},
		\german{Problemlösung}\english{problem-solving},
		\german{Belastbarkeit}\english{ability to work under pressure}
	}

	\emptySeparator

	\experience
	{2016}
	{%
		\german{Gründungsmitglied}%
		\english{Founding member}%
	}
	{FM Studio}
	{\german{Wien}\english{Vienna}}
	{2012}
	{
		\german{
			Location Sound, Foley, Audio-Postproduktion \& Set Design, u.A. für:
		}
		\english{
			Location sound, foley, audio post-production \& set design
			for various projects including:
		}
		\begin{itemize}
			\item History of Now
			      \german{(\textit{Kinofilm, 2016})}%
			      \english{(\textit{feature film, 2016})}%
			\item Herr Tischbein: Wo sie recht hat, hat sie recht
			      \german{(\textit{Musikvideo, 2013})}%
			      \english{(\textit{music video, 2013})}%
			\item Sraia
			      \german{(\textit{Kurzfilm, 2012})}%
			      \english{(\textit{short film, 2012})}%
		\end{itemize}
		\smallskip
	}
	{
		\german{Kommunikation}\english{communication},
		\german{Organisation}\english{organisation},
		\german{Führungsqualität}\english{leadership}
	}
\end{experiences}